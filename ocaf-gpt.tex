% Created 2024-01-16 Tue 11:52
% Intended LaTeX compiler: xelatex
\documentclass[11pt]{article}
\usepackage{graphicx}
\usepackage{longtable}
\usepackage{wrapfig}
\usepackage{rotating}
\usepackage[normalem]{ulem}
\usepackage{amsmath}
\usepackage{amssymb}
\usepackage{capt-of}
\usepackage{hyperref}
\usepackage{xeCJK}
\setCJKmainfont{Heiti SC}
\author{Chen Chen}
\date{\today}
\title{OCAF GPT}
\hypersetup{
 pdfauthor={Chen Chen},
 pdftitle={OCAF GPT},
 pdfkeywords={},
 pdfsubject={},
 pdfcreator={Emacs 29.1 (Org mode 9.7)}, 
 pdflang={English}}
\begin{document}

\maketitle
\tableofcontents

\section{Introduction}
\label{sec:org174545e}

\subsection{Overview}
\label{sec:orge8924ed}

Open CASCADE Technology (OCCT) 是一个开源的软件开发平台,用于三维 CAD、CAM、CAE 系统的开发。它提供了广泛的功能,涵盖了几何建模、图形可视化、数据交换和更多方面。

Open CASCADE Application Framework (OCAF) 是 OCCT 中的一个重要模块。OCAF 是一个应用程序框架,用于简化复杂工程图形应用程序的开发。它提供了一种有效的方式来组织、存储、检索和操作复杂的工程数据。OCAF 特别适合于需要处理复杂的装配结构、历史记录、参数化设计等场景的应用程序。
\subsection{OCAF 的主要功能}
\label{sec:org5721950}

\begin{itemize}
\item 数据管理
\begin{itemize}
\item OCAF 提供了一套工具来有效地管理和组织数据。
\item 这包括用于创建、管理和修改数据结构的 API
\end{itemize}
\item 历史记录和撤销/重做机制
\begin{itemize}
\item OCAF支持记录用户的操作历史,使得可以方便地实现撤销和重做功能。
\end{itemize}
\item 属性和关系管理
\begin{itemize}
\item OCAF允许开发者为数据元素定义属性(如颜色材料等),并管理数据元素间的关系。
\end{itemize}
\item 事务管理
\begin{itemize}
\item OCAF支持事务管理,这对于保证数据的一致性和完整性非常重要。
\end{itemize}
\item 扩展性
\begin{itemize}
\item OCAF设计灵活,易于扩展,开发者可以根据特定应用需求添加新的功能。
\end{itemize}
\end{itemize}

通过 OCAF,开发者可以更专注于应用的核心功能,而不是底层的数据管理和操作。
\subsection{OCAF 的主要 Packages}
\label{sec:org2f1dd28}

下面列出了一些在 OCAF 中常用且重要的 packages:

\begin{itemize}
\item \textbf{TDF (Topological Data Framework)}
\begin{itemize}
\item 用于管理和存储拓扑数据的结构和信息。
\item TDF 提供了一个层次化的数据组织方式,通过 \texttt{Label}, \texttt{Attribute} 等来存储和管理数据
\end{itemize}
\item \textbf{TDocStd (Document Standard)}
\begin{itemize}
\item \texttt{TDocStd} 提供了创建、管理和保存文档的功能
\item 一个文档可以包含一个或多个 TDF 数据结构
\end{itemize}
\item \textbf{XCAF (eXtended CA Framework)}
\begin{itemize}
\item 用于更高级的 CAD 数据处理,如装配结构、颜色和层次信息。
\item XCAF 扩展了 OCAF 的功能,使其能够处理更复杂的 CAD 模型和数据
\end{itemize}
\item \textbf{TNaming (Naming)}
\begin{itemize}
\item 提供了一个命名服务,用于在模型中标识和追踪对象。
\item \texttt{TNaming} 使得在模型变更过程中可以保持对特定对象的引用。
\end{itemize}
\item \textbf{TPrsStd (Presentation Standard)}
\begin{itemize}
\item 用于关联数据模型和其图形表示
\item \texttt{TPrsStd} 允许开发者定义如何将模型数据转换为可视化的图形表示
\end{itemize}
\item \textbf{TDataStd (Data Standard)}
\begin{itemize}
\item 包含了一系列的 Attribute 类型,如字符串、整数、实数、枚举类型等。
\item TDataStd 提供了基础的数据类型,用于存储和处理常规属性。
\end{itemize}
\item \textbf{AppStdL (Application Standard Library)}
\begin{itemize}
\item 提供了一组标准的应用程序功能和服务,如历史管理、撤销/重做机制等。
\end{itemize}
\item \textbf{BinTObj}
\begin{itemize}
\item 用于持久化存储和加载 OCAF 对象的包。
\item 支持二进制格式,适用于大型数据集。
\end{itemize}
\item \textbf{XmlTObj}
\begin{itemize}
\item 类似于 BinTObj,但用于处理基于 XML 的持久化存储和加载。
\end{itemize}
\end{itemize}
\subsection{如何学习掌握 OCCT 及 OCAF}
\label{sec:orgfe9ebf6}

\begin{itemize}
\item 基础了解
\begin{itemize}
\item 了解 OCCT 的基本概念,包括主要组件、功能和应用场景
\item 熟悉 OCAF 模块的基本概念,如 Label, Tag, Attribute 等。
\end{itemize}
\item 阅读官方文档
\item 学习示例代码
\begin{itemize}
\item 查看和分析 OCCT 提供的示例代码,尤其是设计 OCAF 的示例。
\item 通过理解代码,你可以更好地了解如何在实际项目中使用 OCAF 模块。
\end{itemize}
\item 小型项目实践
\begin{itemize}
\item 开始一个小型的项目,使用 OCAF 模块来实现一些基本功能。比如一个简单的 CAD 工具或任何需要数据组织和管理的应用。
\item 在实践中尝试创建、修改和管理 Label、Tag 和 Attribute,以及处理事务和历史记录。
\end{itemize}
\item 深入学习高级特性
\begin{itemize}
\item 当你对基本功能有一定理解后,开始学习 OCAF 的高级特性,如复杂的数据关系管理、历史版本控制、自定义属性类型等。
\end{itemize}
\item 参与社区讨论
\item 查阅相关书籍和资源
\item 实际项目经验
\end{itemize}
\section{TDF Package}
\label{sec:orga2b226f}

\subsection{Overview}
\label{sec:orgfd9993a}

TDF (Topology Data Framework) 是 OCAF 的核心组件,用于管理和组织复杂的工程数据(其中拓扑数据是几何建模的基础)。TDF 提供了一个结构化的方式来存储和操作与拓扑相关的信息,如点、线、面、实体等几何元素及其之间的关系。
\subsection{TDF 的功能与职责}
\label{sec:org6fa67a5}

\begin{itemize}
\item 数据组织

TDF 提供了一种层次化的数据结构,使得对复杂拓扑数据的管理和访问更加直观灵活。

\item 事务管理

通过 TDF,可以实现对拓扑数据的事务管理,支持撤销/重做操作,保证数据一致性。

\item 属性管理

TDF 允许为拓扑元素附加属性(颜色材料等),并管理这些属性。

\item 关系管理

TDF 支持管理拓扑元素之间的关系,如约束、连接等。

\item 版本控制

TDF 支持数据的版本控制,这对于跟踪数据的历史变更非常有用。

\item 灵活性和扩展性

TDF 设计灵活,易于扩展,可以根据特定的应用需求进行定制。
\end{itemize}
\subsection{TDF 的主要接口及功能}
\label{sec:org0c44dc7}

\begin{itemize}
\item \texttt{TDF\_Label} class

\textbf{功能}: 代表数据结构中的一个节点,可包含多个 sub-Labels 和 Attribute。

\textbf{主要接口}: \texttt{FindChild}, \texttt{NewChild}, \texttt{HashAttribute}, \texttt{AddAttribute}, \texttt{FindAttribute}, \texttt{ForEach} 等。

\begin{itemize}
\item \texttt{FindChild} 查找或创建 sub-label
\item \texttt{NewChild} 创建一个新的 sub-label
\item \texttt{HasAttribute} 检查是否存在特定类型的 Attribute
\item \texttt{AddAttribute} 添加一个新的 Attribute
\item \texttt{FindAttribute} 查找特定类型的 Attribute
\end{itemize}

\item \texttt{TDF\_Attribute} class

\textbf{功能}: 附加在 Label 上的数据单元,用于存储特定类型的信息,如几何数据、颜色、文本等。

\textbf{主要接口}: \texttt{Set}, \texttt{Get}, \texttt{NewEmpty}, \texttt{Restore}, \texttt{Paste} 等

\begin{itemize}
\item \texttt{NewEmpty} 创建一个新的空 Attribute 实例
\item \texttt{Restore} 从备份中恢复 Attribute 的状态
\item \texttt{Paste} 复制或转移 Attribute 的内容
\end{itemize}

\item \texttt{TDF\_Data} class

\textbf{功能}: 代表整个数据集合,包含一个或多个 \texttt{TDF\_Label} 树

\textbf{主要接口}:

\begin{itemize}
\item \texttt{Root} 获取数据几何的 root label
\item \texttt{TransactionStart}, \texttt{TransactionCommit} 开始和提交事务
\item \texttt{Undo}, \texttt{Redo} 支持撤销和重做操作
\end{itemize}

\item \texttt{TDF\_TagSource} class

\textbf{功能}: 用于自动生成唯一的 Tag (标签号)。

\textbf{主要接口}: \texttt{NewTag} 生成一个新的唯一 Tag。

\item \texttt{TDF\_RelocationTable} class

\textbf{功能}: 在数据复制和粘贴操作中使用,管理 Label 和 Attribute 之间的关系映射。

\textbf{主要接口}:

\begin{itemize}
\item \texttt{SetRelocation} 设置新旧 Label 或 Attribute 之间的映射
\item \texttt{HasRelocation} 检查是否存在特定的映射
\item \texttt{Relocation} 获取映射的目标
\end{itemize}
\end{itemize}
\section{TDocStd Package}
\label{sec:org1aee66e}

\subsection{Overview}
\label{sec:org675f33d}
\texttt{TDocStd} 主要用于处理和管理文档(Document),这些文档用于存储和组织复杂的 CAD 数据结构。一个文档通常代表一个工程项目或一个 CAD 模型,它包含了所有相关的数据和信息。 \texttt{TDocStd} 提供了一套工具和接口来创建、管理和存储这些文档。
\subsection{TDocStd 的功能与职责}
\label{sec:orgecc3764}

\begin{itemize}
\item 文档管理

\texttt{TDocStd} 提供了创建和管理文档的基本机制。文档可以包含多种类型的数据,如几何形状、装配信息、属性等。

\item 文档结构

文档中的数据通过 OCAF 的 \texttt{TDF\_Label} 结构进行组织。每个文档都有一个 root Label, 从 root Label 开始可以创建一个层次化的数据结构。

\item 事务管理

\texttt{TDocStd} 支持事务管理,允许用户对文档进行修改操作,同时支持 Undo/Redo 功能。这对于保持数据的一致性和完整性至关重要。

\item 存储和加载

\texttt{TDocStd} 提供了将文档保存到文件系统和从文件系统加载文档的功能。支持多种格式,包括自定义格式。

\item 版本控制

文档可以支持版本控制,允许跟踪文档的历史变更。

\item 扩展性

\texttt{TDocStd} 的设计允许开发者根据需要扩展和定制文档的功能,以适应特定的应用需求。
\end{itemize}
\subsection{TDocStd 与 TDF package 的关系}
\label{sec:org10a0a57}

\begin{itemize}
\item \texttt{TDocStd} 依赖于 \texttt{TDF} 来组织文档内的数据。

每个 \texttt{TDocStd\_Document} 包含一个根 \texttt{TDF\_Label}, 这个 root label 是文档所有数据的起点。通过 root label, 可以访问和操作文档中的所有数据。

\item 在 TDF 基础上,TDocStd 提供了文档级别的管理,如创建/保存/加载文档、事务处理(Undo/Redo)等。
\end{itemize}
\subsection{TDocStd 的主要接口及功能}
\label{sec:orgff90817}

\begin{itemize}
\item \texttt{TDocStd\_Document} class

\textbf{功能}: 代表一个文档,是管理和组织 CAD 数据的主要实体。

\textbf{主要接口}:

\begin{itemize}
\item \texttt{NewCommand()} 开始一个新的命令或操作
\item \texttt{CommitCommand()} 提交当前命令,使其更改称为文档的一部分
\item \texttt{Undo()}, \texttt{Redo()} 撤销和重做
\item \texttt{Save()}, \texttt{Open()} 文档的存储和加载
\end{itemize}

\item \texttt{TDocStd\_Application} class

\textbf{功能}: 处理文档的创建、加载和保存,管理文档集合。

\textbf{主要接口}:

\begin{itemize}
\item \texttt{NewDocument()} 创建一个新的文档
\item \texttt{SaveAs()}, \texttt{Open()} 保存和打开文档
\item \texttt{GetFormats()} 获取支持的文档格式列表
\item \texttt{Close()} 关闭文档
\end{itemize}
\end{itemize}


\begin{itemize}
\item \texttt{TDocStd\_Owner} class

\textbf{功能}: 作为文档所有者的角色,管理文档的状态和事务。

\textbf{主要接口}:
\begin{itemize}
\item \texttt{SetDocument} 设置或关联文档
\item \texttt{BeforeUndo}, \texttt{AfterUndo} 撤销操作前后的处理函数。
\end{itemize}
\end{itemize}
\section{XCAF Package (属于 DataExchange Module)}
\label{sec:org2cb5ac6}

\subsection{Overview}
\label{sec:orgf6420c9}

XCAF (eXtended CA Framework) 用于处理更高级别的 CAD 数据,尤其是那些涉及到复杂装配结构的数据。XCAF 提供了一些列工具和接口,用于管理和操作包括颜色、材料、元数据、层级关系等在内的复杂 CAD 模型数据。
\subsection{XCAF 主要功能与职责}
\label{sec:org387c4df}

\begin{itemize}
\item 复杂装配结构管理

XCAF 提供了工具来创建和管理复杂的 CAD 装配结构,包括定义装配体、子装配体和零件之间的层级关系。

\item 颜色和图层管理

支持为模型的不同部分指定颜色和图层,帮助改善模型的可视化和组织。

\item 高级属性管理

XCAF 允许为模型元素添加和管理高级属性,如材料属性、PMI(产品和制造信息)、注释和元数据。

\item 形状标识和追踪

提供工具来唯一标识和追踪模型中的形状,尤其在模型的变更或更新过程中,保持对特定形状的引用。

\item 数据交换支持

支持与其他 CAD 系统间的数据交换,特别是在处理 STEP 和 IGES 文件格式时,能够导入和导出中配信息和属性。

\item 扩展性和定制

XCAF 设计灵活,可以根据特定应用需求进行扩展和定制。
\end{itemize}
\subsection{XCAF 的主要接口与功能}
\label{sec:orgcad4401}

\begin{itemize}
\item \texttt{XCAFDoc\_ShapeTool} class

\textbf{功能}: 用于管理装配结构和形状。

\textbf{主要接口}:

\begin{itemize}
\item \texttt{GetShape} 获取形状
\item \texttt{AddShape} 添加新的形状
\item \texttt{GetSubShapes}, \texttt{GetSubShapeExt} 获取子形状
\item \texttt{GetAssembly} 获取装配体
\end{itemize}

\item \texttt{XCAFDoc\_ColorTool} class

\textbf{功能}: 管理颜色属性

\textbf{主要接口}:

\begin{itemize}
\item \texttt{SetColor} 为形状设置颜色
\item \texttt{GetColor} 获取形状的颜色
\item \texttt{RemoveColor} 移除形状的颜色
\end{itemize}

\item \texttt{XCAFDoc\_LayerTool} class

\textbf{功能}: 管理图层属性

\textbf{主要接口}:

\begin{itemize}
\item \texttt{SetLayer} 为形状设置图层
\item \texttt{GetLayers} 获取形状的图层
\end{itemize}

\item \texttt{XCAFDoc\_MaterialTool} class

\textbf{功能}: 管理材料属性

\textbf{主要接口}:

\begin{itemize}
\item \texttt{SetMaterial} 为形状设置材料
\item \texttt{GetMaterial} 获取形状的材料
\end{itemize}

\item \texttt{XCAFDoc\_DatumTool}, \texttt{XCAFDoc\_DimTolTool} classes

\textbf{功能}: 管理标注和公差。

\textbf{主要接口}:

\begin{itemize}
\item \texttt{AddDatum}, \texttt{AddDimTol} 添加新的标注或公差
\item \texttt{GetDatum}, \texttt{GetDimTol} 获取标注或公差
\end{itemize}

\item \texttt{XCAFDoc\_AreaStyleTool} class

\textbf{功能}: 管理区域样式

\textbf{主要接口}

\begin{itemize}
\item \texttt{SetAreaStyle} 为形状设置区域样式
\item \texttt{GetAreaStyle} 获取形状的区域样式
\end{itemize}
\end{itemize}
\section{TNaming package}
\label{sec:orgfd61bb6}

\subsection{Overview}
\label{sec:org19e1c4d}

\texttt{TNaming} 提供了命名服务,以便在复杂的 CAD 模型和数据结构中标识和追踪对象。这对于在模型变更过程中保持对特定对象的引用非常重要。
\subsection{TNaming 的主要功能与职责}
\label{sec:orgb1f2909}

\begin{itemize}
\item 对象标识和追踪

\texttt{TNaming} 允许用户为模型中的对象(如形状、特征等)赋予唯一的名称,从而在整个模型的生命周期中追踪和引用这些对象。

\item 历史追踪

支持记录和跟踪对象随时间的变化。这使得即使在模型被修改或更新后,也能够识别和访问原始对象。

\item 版本控制

\texttt{TNaming} 提供了一种机制来处理模型中对象的版本控制,保证在多次修改和迭代中对象的一致性。

\item 复杂操作支持

对于复杂的操作(如布尔运算、分割、修剪等), \texttt{TNaming} 能够帮助保持对影响的对象的引用,确保数据的准确性和完整性。

\item 与 TDF 协同工作

\texttt{TNaming} 与 \texttt{TDF} 紧密协作,利用 \texttt{TDF\_Label} 和 \texttt{TDF\_Attribute} 来存储和管理命名信息。

\item 撤销/重做机制支持

支持与 OCAF 的撤销/重做机制结合使用,确保在运行这些操作时保持命名信息的一致性。
\end{itemize}
\subsection{TNaming 的主要接口与功能}
\label{sec:org35cbe06}

\begin{itemize}
\item \texttt{TNaming\_NamedShape} class

\textbf{功能}: 用于关联形状(Shape)与名称,实现形状的命名和追踪。

\textbf{主要接口}

\begin{itemize}
\item \texttt{Get} 获取与名称关联的形状
\item \texttt{Set} 设置或更新名称与形状的关联
\end{itemize}

\item \texttt{TNaming\_Builder} class

\textbf{功能}: 用于构建和修改命名关系

\textbf{主要接口}

\begin{itemize}
\item \texttt{Select} 为给定的形状选择或创建一个名称
\item \texttt{Generate} 生成一个新的名称
\end{itemize}

\item \texttt{TNaming\_Tool} class

\textbf{功能}: 提供一系列静态方法来操作和查询命名信息

\textbf{主要接口}

\begin{itemize}
\item \texttt{GetShape} 根据名称获取形状
\item \texttt{GetLabel} 获取与特定形状关联的标签
\end{itemize}

\item \texttt{TNaming\_Naming} class

\textbf{功能}: 存储和管理命名操作的历史记录。

\textbf{主要接口}

\begin{itemize}
\item \texttt{GetName} 获取命名操作的名称
\item \texttt{GetShapes} 获取命名操作影响的形状列表
\end{itemize}

\item \texttt{TNaming\_NamingTool} class

\textbf{功能}: 提供用于执行复杂命名操作的高级方法

\textbf{主要接口}

\begin{itemize}
\item \texttt{Solve} 解决命名冲突
\item \texttt{LoadNamedShapes} 加载命名形状
\end{itemize}
\end{itemize}
\section{TPrsStd package}
\label{sec:orgabf9f64}

\subsection{Overview}
\label{sec:org01b67ae}

\texttt{TPrsStd} package 用于将工程数据(如存储在 OCAF 文档中的数据)与其图形表示相关联。它为开发者提供了一系列工具和接口,以便在图形界面中展示和交互复杂的工程模型。
\subsection{TPrsStd 主要功能与职责}
\label{sec:orgd2c40df}

\begin{itemize}
\item 图形表示管理

\texttt{TPrsStd} 使得开发者可以将工程数据(如形状、属性等)与其在图形用户界面中的视觉表示相关联。这包括形状的渲染、颜色、纹理等。

\item 交互和选择支持

提供了工具来支持用户与图形表示的交互,包括选择、高亮显示和编辑操作。

\item 属性与视觉同步

确保工程数据的更改能够实时反映在图形表示上,例如当形状发生变化时,其视觉表示也会相应更新。

\item 高级显示功能

支持高级的显示功能,如透明度、阴影和纹理映射,使得工程模型的视觉表示更加逼真和详细。

\item 自定义显示属性

允许开发者定义自己的显示属性和表示方式,以满足特定应用的需求。

\item 与 OCAF 结合使用

\texttt{TPrsStd} 与 OCAF的其他组件(如 \texttt{TDF\_Label}, \texttt{TDF\_Attribute})紧密集成,使得开发者可以方便地管理和同步数据与其图形表示。

\item 支持多种渲染引擎

可以与 OCCT 提供的不同渲染引擎(如OpenGL)协同工作,提供高质量的图形输出。
\end{itemize}
\subsection{TPrsStd 的主要接口与功能}
\label{sec:orgc664034}

\begin{itemize}
\item \texttt{TPrsStd\_AISPresentation} class

\textbf{功能}: 用于管理工程数据的图形表示,如形状在图形界面中的显示。

\textbf{主要接口}

\begin{itemize}
\item \texttt{SetDriver} 设置用于显示的驱动程序。
\item \texttt{Update} 更新图形表示以反映数据的更改。
\item \texttt{Display}, \texttt{Erase} 控制对象的显示和隐藏。
\end{itemize}

\item \texttt{TPrsStd\_AISViewer} class

\textbf{功能}: 提供一个视图环境,用于显示和管理多个图形表示。

\textbf{主要接口}

\begin{itemize}
\item \texttt{Update} 更新视图中的所有表示
\item \texttt{SetView} 设置或更改关联的视图
\end{itemize}

\item \texttt{TPrsStd\_Presentation} class

\textbf{功能}: 作为数据和其图形表示之间的桥梁。

\textbf{主要接口}

\begin{itemize}
\item \texttt{Set} 关联数据和图形表示。
\item \texttt{Get} 获取与数据关联的图形表示。
\end{itemize}

\item \texttt{TPrsStd\_Driver} class

\textbf{功能}: 为具体的数据类型提供图形表示的生成和更新逻辑。

\textbf{主要接口}

\begin{itemize}
\item \texttt{Update} 根据数据更新图形表示。
\item \texttt{Create} 根据给定的数据创建新的图形表示。
\end{itemize}
\end{itemize}
\section{TDataStd package}
\label{sec:orgbe8d23c}

\subsection{Overview}
\label{sec:org9937773}

\texttt{TDataStd} 主要提供了一系列标准的数据属性(Attributes),这些属性可以附加到 OCAF 文档中的 Labels 上,用于存储和管理各种类型的数据。
\subsection{TDataStd 主要功能与职责}
\label{sec:org01e6f62}

\begin{itemize}
\item 基本数据类型的管理

\texttt{TDataStd} 提供了用于存储基本数据类型(如字符串、整数、实数、布尔值等)的属性。这些属性用于存储和检索与标签相关联的基本信息。

\item 集合和列表的管理

提供了管理数据集合(如数组、列表)的属性,用于存储多个数据项。

\item 命名和标识符管理

支持为标签分配名称和标识符,方便数据的识别与引用。

\item 枚举和状态管理

提供了用于管理枚举值和状态的属性,可以用于表示有限的选择集或状态机。

\item 文档的元数据管理

支持存储文档级别的元数据,如作者、版本信息、注释等。

\item 与 \texttt{TDF\_Label} 结合使用

\texttt{TDataStd} 的属性与 \texttt{TDF\_Label} 紧密集成,使得数据可以方便地附加到标签上,并在 OCAF 文档的层次化结构中进行管理。
\end{itemize}
\subsection{TDataStd 的主要接口与功能}
\label{sec:org7157bb1}

\begin{itemize}
\item \texttt{TDataStd\_Integer}
\begin{itemize}
\item 功能: 用于存储和管理整数值
\item 主要接口: \texttt{Set}, \texttt{Get}
\end{itemize}
\item \texttt{TDataStd\_Real}
\begin{itemize}
\item 功能: 用于存储和管理实数值
\item 主要接口: \texttt{Set}, \texttt{Get}
\end{itemize}
\item \texttt{TDataStd\_String}
\begin{itemize}
\item 功能: 用于存储和管理字符串
\item 主要接口: \texttt{Set}, \texttt{Get}
\end{itemize}
\item \texttt{TDataStd\_UAttribute}
\begin{itemize}
\item 功能: 作为用户自定义数据的基类,可以派生出用于存储特定类型数据的类。
\item 主要接口: \texttt{SetID} 设置属性的唯一标识符
\end{itemize}
\item \texttt{TDataStd\_Name}
\begin{itemize}
\item 功能: 用于存储和管理对象的名称
\item 主要接口: \texttt{Set}, \texttt{Get}
\end{itemize}
\item \texttt{TDataStd\_Boolean}
\begin{itemize}
\item 功能: 用于存储和管理布尔值
\item 主要接口: \texttt{Set}, \texttt{Get}
\end{itemize}
\item \texttt{TDataStd\_Enum}
\begin{itemize}
\item 功能: 用于存储和管理枚举值
\item 主要接口: \texttt{Set}, \texttt{Get}
\end{itemize}
\end{itemize}
\section{AppStdL package}
\label{sec:org77854b6}

\subsection{Overview}
\label{sec:orgdf5016e}

\texttt{AppStdL} (Application Standard Library) 包提供了标准化的应用程序级功能和服务,旨在简化复杂工程应用程序的开发过程。
\subsection{AppStdL 的功能与职责}
\label{sec:org4e4179a}

\begin{itemize}
\item 标准文档管理

提供标准的文档管理功能,包括文档的创建、打开、保存、关闭等操作。这些操作通常是大多数工程应用程序的基础。

\item 事务管理

支持事务处理机制,允许对文档进行修改操作,并支持 Undo/Redo 功能。这对于保证数据的一致性和完整性非常重要。

\item 用户界面交互

提供与用户界面交互相关的标准功能,如命令处理、事件响应等。

\item 数据存储和加载

支持标准化的数据存储和加载机制,包括对不同格式的文件的处理,如 XML、二进制等。

\item 应用程序配置

提供应用程序配置的管理,允许存储和检索应用程序设置和参数。

\item 资源管理

管理应用程序所需的资源,如图像、图标、样式表等。
\end{itemize}
\subsection{AppStdL 主要接口与功能}
\label{sec:org129e249}

\begin{itemize}
\item \texttt{AppStd\_Application} class

\textbf{功能}: 作为应用程序的核心类,负责管理文档和用户界面的交互。

\textbf{主要接口}: \texttt{NewDocument}, \texttt{OpenDocument}, \texttt{SaveDocument}, \texttt{CloseDocument}

\item \texttt{AppStd\_Document} class

功能: 代表应用程序中的单个文档,负责管理文档中的数据和事务。

主要接口:

\begin{itemize}
\item \texttt{InitNew} 初始化一个新的文档
\item \texttt{Open}, \texttt{Save} 打开/保存文档
\item \texttt{Undo}, \texttt{Redo} 撤销/重做操作
\end{itemize}

\item \texttt{AppStd\_Command} class

功能: 封装用户界面中的命令操作,用于处理事件和执行特定的动作。

主要接口

\begin{itemize}
\item \texttt{Execute} 执行命令
\item \texttt{Undo}, \texttt{Redo} 命令的撤销和重做
\end{itemize}

\item \texttt{AppStd\_Preferences} class

功能: 管理应用程序的配置和偏好设置。

主要接口:

\begin{itemize}
\item \texttt{ReadPreferences} 读取配置设置
\item \texttt{WritePreferences} 写入配置设置
\end{itemize}
\end{itemize}
\section{BinTObj package}
\label{sec:orgc8f49d0}

\section{XmlTObj packages}
\label{sec:org31bca08}
\end{document}
